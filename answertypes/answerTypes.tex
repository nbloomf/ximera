\documentclass{ximera}
\title{\LaTeX\ Commands}
\begin{document}

\begin{abstract}
  This section explores the \LaTeX\ commands Ximera supports. 
\end{abstract}

\maketitle

Ximera supports many commands.


\section{Basic Function Answer Type}

\begin{verbatim}
         $3\times 2 = $ \answer{6}
\end{verbatim}

Will produce the question:

\begin{question}
  $3\times 2 = $ \answer{6}
\end{question}


In addition to numerical answers, we also support elementary functions:

\begin{verbatim}
    \begin{question}
         $ \frac{\partial}{\partial x} x^2\sin(y) = $ \answer{2xsin(y)}
    \end{question}
\end{verbatim}

Produces:

\begin{question}
  $ \frac{\partial}{\partial x} x^2\sin(y) = $ \answer{2xsin(y)}
\end{question}

\begin{remark}
Under the hood, Ximera is parsing the user input, producing a
function, and checking the user input function against ``answer'' at
$100$ different complex numbers, and seeing if the results are
``reasonably'' close to each other.  We compare the complex extensions
of these functions to circumvent domain issues.
\end{remark}

We also support matrices of expressions in answers.

\begin{verbatim}
\begin{question}
Enter the matrix  \(\begin{bmatrix} x & y \\ xy & z+1 \end{bmatrix}\)
    \begin{matrixAnswer}
	    correctMatrix = [['x','y'],['xy','z+1']]
    \end{matrixAnswer}
\end{question}
\end{verbatim}

\begin{question}
  Enter the matrix  \(\begin{bmatrix} x & y \\ xy & z+1 \end{bmatrix}\)
  \begin{matrixAnswer}
    correctMatrix = [['x','y'],['xy','z+1']]
   % This doesn't seem to compile.
  \end{matrixAnswer}
\end{question}

\begin{remark}
  The plus and minus muttons add and subtract columns or rows.  
\end{remark}



\section{Multiple Choice Answer Type}

\begin{verbatim}
\begin{question}
Which of the following functions has a graph which is a parabola?
  \begin{multipleChoice}
    \choice[correct]{$y=x^2+3x-3$}
    \choice{$y = \frac{1}{x+2}$}
    \choice{$y=3x+1$}
  \end{multipleChoice}
\end{question}
\end{verbatim}

Produces:

\begin{question}
  Which of the following functions has a graph which is a parabola?
  \begin{multipleChoice}
    \choice[correct]{$y=x^2+3x-3$}
    \choice{$y = \frac{1}{x+2}$}
    \choice{$y=3x+1$}
  \end{multipleChoice}
\end{question}

\begin{remark}
  Multiple choice questions are automatically shuffled.
\end{remark}



\section{Free-response}

The free response environment gives students access to a \LaTeX\ editor. 

\begin{freeResponse}
  This is the model solution.
\end{freeResponse}

\begin{remark}
Clicking on \verb!View model solution! shows the user
whatever you typed in the  \verb!freeResponse! environment.
\end{remark}



\section{Explanation}

\begin{explanation}
This is an explanation.
\end{explanation}



\section{Example}

\begin{example}
This is an example.
\end{example}



\section{Prompt}

\begin{prompt}
This is a prompt.
\end{prompt}



\section{Code}

\begin{code}
This is some code.
\end{code}



\section{Python}

\begin{python}
This is some python
\end{python}



\section{Parts}

\begin{parts}
\item Part 1
\item Part 2
\end{parts}



\section{Shuffle}

\begin{shuffle}
This is a shuffle environment. What goes here?
\end{shuffle}



\section{Solution}

\begin{solution}
What is this?
\end{solution}



\section{Interactive}

\begin{interactive}
What is this?
\end{interactive}



\section{Feedback}

\begin{feedback}
What is this?
\end{feedback}



\section{Dialogue}

\begin{dialogue}
\item[Nathan] What is this?
\item[Hans] This is for showing a conversation.
\end{dialogue}



\section{Theorems}

\begin{theorem}
What is this? (there are other theorems)
\end{theorem}



\section{Word Choice}

\wordChoice[wrong 1, wrong 2]{right}


\section{Linking}

You should see some links here.

\video{https://www.youtube.com/watch?v=dQw4w9WgXcQ}
\youtube{https://www.youtube.com/watch?v=dQw4w9WgXcQ}
\link{https://www.youtube.com/watch?v=dQw4w9WgXcQ}
\link{}



\section{Prerequisites}

\prerequisites{This is a prerequisite}



\section{Outcome}

\outcome{This is an outcome.}



\section{Xarma Boost}

\begin{xarmaBoost}
no idea here.

\begin{question}
Maybe questions go here? \answer{Let's find out.}
\end{question}
\end{xarmaBoost}

\end{document}
