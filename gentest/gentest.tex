\documentclass{ximera}

\begin{document}

\title{A Generator Test}
\begin{abstract}
This is a test.
\end{abstract}
\maketitle

In this example, we will construct a worksheet of 10 basic quadratic polynomial factorization problems with small integer roots. In my experience, the best way to do this is to work "backwards"; start with the kind of answer we want and think carefully about the skills needed to get there.


Rather than starting with 10 polynomials, let's start with 10 pairs of roots. We can represent these in feivel as a list of lists of integers. Let's say each root is an integer between -6 and 6, not including 0 (since allowing 0 as a root yields a fundamentally different kind of problem).
\begin{center}
\verb|define {int} @roots := Range(-6;6) \\ {0}|
\end{center}

Next we can choose all pairs $(a,b)$ of possible roots such that $a+b \neq 0$, so we don't end up with a difference of squares (that's a different problem). We can also force $a < b$ to avoid getting duplicates. (Here we are using list builder notation, similar to set builder notation.)
\begin{center}
\verb|define {{int}} @pairs := Build({@a;@b}; int @a <- @roots;|
\verb|int @b <- @roots; NotEqual(int; @a+@b; 0); LT(int; @a; @b))|
\end{center}

Finally, we choose 10 pairs at random.
\begin{center}
\verb|define {{int}} @xss := Choose(10; @pairs)|
\end{center}

Now we will iterate through this list of pairs, formatting a polynomial problem for each one.

\begin{enumerate}
% Extract the roots from each pair
  % Define our polynomial (x-a)(x-b) and format it nicely
  % Define the solution and format it nicely
  % Display the problem
  \item $x^2 - 2x - 15 = (x + 3)(x - 5)$
% Extract the roots from each pair
  % Define our polynomial (x-a)(x-b) and format it nicely
  % Define the solution and format it nicely
  % Display the problem
  \item $x^2 + 6x + 8 = (x + 4)(x + 2)$
% Extract the roots from each pair
  % Define our polynomial (x-a)(x-b) and format it nicely
  % Define the solution and format it nicely
  % Display the problem
  \item $x^2 + 5x + 4 = (x + 4)(x + 1)$
% Extract the roots from each pair
  % Define our polynomial (x-a)(x-b) and format it nicely
  % Define the solution and format it nicely
  % Display the problem
  \item $x^2 - 3x - 10 = (x + 2)(x - 5)$
% Extract the roots from each pair
  % Define our polynomial (x-a)(x-b) and format it nicely
  % Define the solution and format it nicely
  % Display the problem
  \item $x^2 + 3x - 10 = (x + 5)(x - 2)$
% Extract the roots from each pair
  % Define our polynomial (x-a)(x-b) and format it nicely
  % Define the solution and format it nicely
  % Display the problem
  \item $x^2 - 5x + 6 = (x - 2)(x - 3)$
% Extract the roots from each pair
  % Define our polynomial (x-a)(x-b) and format it nicely
  % Define the solution and format it nicely
  % Display the problem
  \item $x^2 + 4x - 12 = (x + 6)(x - 2)$
% Extract the roots from each pair
  % Define our polynomial (x-a)(x-b) and format it nicely
  % Define the solution and format it nicely
  % Display the problem
  \item $x^2 - 2x - 24 = (x + 4)(x - 6)$
% Extract the roots from each pair
  % Define our polynomial (x-a)(x-b) and format it nicely
  % Define the solution and format it nicely
  % Display the problem
  \item $x^2 - 3x - 18 = (x + 3)(x - 6)$
% Extract the roots from each pair
  % Define our polynomial (x-a)(x-b) and format it nicely
  % Define the solution and format it nicely
  % Display the problem
  \item $x^2 - 8x + 12 = (x - 2)(x - 6)$
\end{enumerate}

\end{document}
